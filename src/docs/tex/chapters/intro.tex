

\section{About}
\subsection{General}
It is the year 2084. You control a secret organisation charged with defending Earth from a brutal alien enemy. Build up your bases, prepare your team, and dive head-first into the fast and flowing turn-based combat.\\
\\
UFO: ALIEN INVASION is a squad-based tactical strategy game in the tradition of the old X-COM PC games, but with a twist. Our game combines military realism with hard science-fiction and the weirdness of an alien invasion. The carefully constructed turn-based system gives you pin-point control of your squad while maintaining a sense of pace and danger.\\
\\
Over the long term you will need to conduct research into the alien threat to figure out their mysterious goals and use their powerful weapons for your own ends. You will produce unique items and use them in combat against your enemies. If you like, you can even use them against your friends with our multiplayer functionality.\\
\\
'UFO: Alien Invasion'. Endless hours of gameplay -- absolutely free.\\
\\
The game takes a lot of inspiration from the X-COM series by Mythos and Microprose. However, it's neither a sequel nor a remake of any X-COM or other commercial title. What we as a team wanted to make is a brand new experience that tries to surpass the quality of games from 1992, rather than simply recreate them with flashier graphics.\\
\\
We also believe that open-source projects don't need to be disorganised or badly-managed. We work together in a friendly and professional way, with a clear vision for a game that we know is worth playing.

\subsection{Gameplay}
Like the original X-COM games, UFO:AI has two main modes of play: Geoscape mode and Tactical mode.\\
\\
In Geoscape mode, the game is about base management and strategy. You manage the activities and finances of PHALANX, controlling bases, installations, aircraft and squads of armed-response troops. You'll research new technologies and use their results in battle against the aliens. You'll launch your interceptor aircraft to shoot down UFOs and dispatch dropships in response to alien activity across the globe. You can build, buy and produce anything you like, as long as your technology level and your budget will allow it.\\
\\
Geoscape mode employs easy-to-use time buttons to control the passage of time however you like, automatically pausing whenever there's an important message for you to see. Research will progress and items will be produced as time passes.\\
\\
In Tactical mode, the game is about taking command of your team in various missions to combat the aliens wherever they might appear. Instead of pushing imperonal armies around on giant maps, you use only the team of soldiers you've assigned to deal with this mission -- the same soldiers which you've managed and equipped to your satisfaction in Geoscape mode. You may find civilians and other bystanders during the mission, being targeted and executed by the aliens, or just getting in the way while you're trying to protect them. It's a dangerous world, and some of your men will die.\\
\\
Tactical mode uses a turn-based system, where your team and the aliens take turns to make moves. During your turn you can order your troops to move around, fire their weapons, throw grenades or use other equipment, etc. Each soldier gets a certain number of Time Units (TUs), representing the total time they have to act during the current turn. All actions in Tactical mode require Time Units to perform. Once a soldier is out of Time Units, he or she can't do anything more until the next turn.\\
\\
Your mission objectives will vary for each mission, and there are many to perform as the aliens' terrifying plot unfolds. You'll have to watch your back, be quick on your feet, and take the fight to them.\\
\\
If you don't, humanity is doomed.

\subsection{Game Engine}
The game engine is based on a heavily modified version of ID's Quake2 Engine. This doesn't mean that 'UFO: Alien Invasion' is a modification or even a total conversion of Quake 2. It is a stand-alone game and doesn't require Quake 2 or any other program to run. All you need to play UFO:AI is a computer running Microsoft Windows or a supported version of Linux, and the installer from our website\footnote{\url{http://ufoai.sf.net}}.\\
\\
Our updated engine has modern OpenGL graphics and special effects, increased texture resolution, hardware-accelerated clipping of map layers for toggling between levels, a new animation system for player models, powerful artificial intelligence, and many other exciting features.

\section{Free games / the community}
\subsection{Contribute}
This game is brought to you by the UFO:AI Development team and its countless contributors. All of them share at least one thought: to make UFO:AI a great free \footnote{free as in freedom} game. Besides detailed legal implications, mentioned in the following section and given in the appendix, most of all this means that every piece of code used to create this game is publicly available. Even more: you are free - even wanted - to change everything you want by yourself whenever you feel you can help making UFO:AI a better way to waste time. This may start with typos or end with complete mods or patches - it's up to you. With UFO:AI being an open-source development by a bunch of non profit orientated people this does also mean there is no big company in the back to pay for extensive testing, balancing or hardwarechecking. So whenever you encounter a bug, a hardware incompatibility or any other problem it would be a fair gesture to give something back to the community - even a carefully filled out bugreport \footnote{\url{http://sourceforge.net/tracker/?atid=805242&group_id=157793&func=browse}} helps a lot. So we hope to do our little share to promote free software and build up a productive open-source gaming community. And no matter what kind of skills you call your own, if you are a coder, 2D or 3D artist, map-designer, even film-script writer, musician, concept-art designer (all of these made UFO:AI what it is today) be assured that there is a project out there waiting for your help - enriching the pool of free software.\\
If you are interested, please also visit the \textit{Contribute} section in our wiki at \url{http://ufoai.sf.net}. You will find a lot of useful information about model and image formats, a lot of tutorials about mapping and so on.

\subsection{Contact / Support}
Support, additional information, FAQs and the forum can be found at \url{http://ufoai.sf.net}.
For a release history, latest releases and bugfixes as well as the bug- and featuretracker\footnote{\url{http://sourceforge.net/tracker/?group_id=157793}} please see our project page at \url{http://www.sourceforge.org}.
Sourceforge also offers you to take a look at our project page (where you find detailed status reports, contribution- and memberlists). In addition to the forum we also host the channel ufo:ai on the freenode\footnote{irc.freenode.org} IRC network. As usual and according to netiquette pleases make sure you try to find solutions for rather trivial problems on your own before asking on the board or on IRC.\\
For interested media we also provide screenshots and offer further support for any planed coverage - feel free to contact us personally by one of the above ways.
